\documentclass{beamer}

\usetheme{Darmstadt}

\usepackage{graphics,xcolor}
\usepackage{stmaryrd,amssymb,amsmath}
\usepackage[most]{tcolorbox}
\newtcolorbox{lol}[1]{colback=red!5!white,colframe=red!75!black,fonttitle=\bfseries,title=#1,breakable}
\usepackage{docmute}
\usepackage{hyperref}
\newcommand{\nat}{\mathbb{N}}
\newcommand{\ints}{\mathbb{Z}}
\newcommand{\intersection}{\ensuremath{\cap}}
\newcommand{\emptyword}{\ensuremath{\epsilon}}
\newcommand{\len}[1]{\ensuremath{|#1|}}
\newcommand{\union}{\ensuremath{\cup}}
\newcommand{\deltahat}{\ensuremath{\widehat{\delta}}}

\title{Applying Dupeire's Local Volatility Model to CVaR minimization}

\author{Balaji}
\date{}
\institute{IISc Bangalore}
\begin{document}
\maketitle
\section{The local volatility model}
\begin{frame}{Stock Price movement}
    \begin{lol}{Equation}
        \begin{equation}
            dS_t = rS_t dt + \sigma(t,S_t)S_t\tilde{dW_t}
        \end{equation}
    \end{lol}
    
    \begin{itemize}
        \item $S_t$ is the stock price at time $t$
        \item $r$ is the risk-free rate
        \item $\sigma(t,S_t)$ is the volatility as a function of the time and stock price.
        \item $\tilde{W_t}$ is the Brownian Motion under the risk-neutral measure.
    \end{itemize}
    
\end{frame}
\begin{frame}
    \frametitle{The Dupire Equation}
    \begin{lol}{Equation}
        \begin{equation}
            \frac{\partial C(T,K)}{\partial T}  + rK\frac{\partial C(T,K)}{\partial K}  = \frac{1}{2}\sigma^2(T,K)K^2\frac{\partial^2 C(T,K)}{\partial K^2}
        \end{equation}
    \end{lol}
    \begin{itemize}
        \item C is the price of the call option
        \item T is the time to maturity
        \item K is the strike price.
    \end{itemize}
    \begin{lol}{Solving For The volatility}
        \begin{equation}
            \sigma(T,K) = \frac{1}{K} \sqrt{\frac{2\partial C/\partial T(T,K) + 2rK\partial C(T,K)/\partial K}{\partial^2C/\partial K^2(T,K)}}
        \end{equation}
    \end{lol}
\end{frame}
\section{Building the Model(To do..)}
\begin{frame}{Steps(Taken from the net)}
     \begin{itemize}
        \item Collect option prices for various strikes and maturities.
        \item Interpolate/Extrapolate the option prices/ Black-Scholes Volatilities to get a smooth volatility surface.
        \item Use the dupeire equation to get the volatility function.
        \item Can calculate prices of other options using finite difference and monte-carlo simulations.
     \end{itemize}
\end{frame}
\begin{frame}{Integrating into our CVaR minimization Problem(Doubtful??)}
    \begin{itemize}
        \item Once we have the volatility function, we can integrate it into our pricing model and get the estimated \textbf{value} of the options at maturity using simulations.
        \item Run the fast gradient descent algorithm.
    \end{itemize}
\end{frame}
\section{References}
\begin{frame}
    \frametitle{References:}
    \begin{itemize}
        \item \href{https://www.csie.ntu.edu.tw/~d00922011/python/cases/LocalVol/DUPIRE_FORMULA.PDF}{\textcolor{blue}{Online Resource}}
        \item Shreve, Stochastic Calculus for Finance II: Continuous-Time Models
    \end{itemize}
\end{frame}
\end{document}